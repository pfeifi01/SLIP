% 2015 | UIBK
% vim:set spell tw=79:

\documentclass[beamer]{uibk}
\title{SLIP}
\subtitle{Serial Line Internet Protocol}
\author{A. Fedrigolli, R. Gritzer, M. Pfeifhofer }
\date{\today}

\hypersetup{colorlinks,
            citecolor=black,
            filecolor=black,
            linkcolor=black,
            urlcolor=black}
\graphicspath{{./media/}}

\usepackage{color}
\usepackage{tabto}
\usepackage{listings}

\definecolor{listinggray}{gray}{0.9}
\definecolor{lbcolor}{rgb}{0.9,0.9,0.9}

\lstset{
    backgroundcolor=\color{lbcolor},
    tabsize=4,
    language=C,
    basicstyle=\scriptsize,
    upquote=true,
    aboveskip={1.5\baselineskip},
    columns=fixed,
    showstringspaces=false,
    extendedchars=false,
    breaklines=true,
    prebreak = \raisebox{0ex}[0ex][0ex]{\ensuremath{\hookleftarrow}},
    frame=single,
    numbers=left,
    showtabs=false,
    showspaces=false,
    showstringspaces=false,
    identifierstyle=\ttfamily,
    keywordstyle=\color[rgb]{0,0,1},
    commentstyle=\color[rgb]{0.026,0.112,0.095},
    stringstyle=\color[rgb]{0.627,0.126,0.941},
    numberstyle=\color[rgb]{0.205, 0.142, 0.73}
}

\begin{document}

\maketitle

\begin{frame}{Problem}
  Wir schreiben das Jahr 1988.

  \includegraphics[scale=0.5]{1988computer.jpg}

\end{frame}

\begin{frame}{Ansatz}
  IP Pakete über die serielle Schnittstelle  übertragen
  \begin{center}
  \includegraphics[scale=0.5]{ansatz.jpg}
  \end{center}
\end{frame}

\begin{frame}{Einordnung}
  \begin{center}
  \includegraphics[width=\textwidth,height=\textheight,keepaspectratio]{layer.jpg}
  \end{center}
\end{frame}

\begin{frame}{Einordnung}
  \begin{center}
  \includegraphics[width=\textwidth,height=\textheight,keepaspectratio]{layer2.jpg}
  \end{center}
\end{frame}

\begin{frame}{Details}
  \newpage
  \begin{center}
  \includegraphics[width=\textwidth,height=\textheight,keepaspectratio]{ip1.png}
  \end{center}
\end{frame}

\begin{frame}{Details}
  \newpage
  \begin{center}
  \includegraphics[width=\textwidth,height=\textheight,keepaspectratio]{ip2.png}
  \end{center}
\end{frame}

\begin{frame}{Details}
  \newpage
  \begin{center}
  \includegraphics[width=\textwidth,height=\textheight,keepaspectratio]{escaping.png}
  \end{center}
\end{frame}

\begin{frame}{Details}
  \newpage
  \begin{center}
  \begin{itemize}
    \item END      \tabto{3cm} 0xC0 (192)
    \item ESC      \tabto{3cm} 0xDB (219)
    \item ESC\_END \tabto{3cm} 0xDC (220)
    \item ESC\_ESC \tabto{3cm} 0xDD (221)
  \end{itemize}
  \end{center}
\end{frame}

\begin{frame}{Details}
  \newpage
  \begin{center}
    \lstinputlisting{src/send.c}
  \end{center}
\end{frame}

\begin{frame}{}
    \begin{block}{Vorteile}
      \begin{itemize}
        \item Sehr einfache Implementierung
        \item Sehr wenig Overhead
      \end{itemize}
    \end{block}

    \begin{alertblock}{Nachteile}
      \begin{itemize}
        \item Steuersignale können Verbindung unterbrechen (z.B. Strg-C)
        \item Keine Fehlererkennung
        \item Übertragungsrate (1,2 kbps - 19,2 kbps)
        \item Keine Meta-Daten übertragbar
      \end{itemize}
    \end{alertblock}

\end{frame}

\begin{frame}{Erweiterungs - Wünsche}
  \begin{itemize}
    \item Fehlerkorrektur
    \item Daten-Komprimierung
    \item Rechner Adressierung
    \item Multi-Protokoll Fähigkeit
  \end{itemize}

\end{frame}

\end{document}
